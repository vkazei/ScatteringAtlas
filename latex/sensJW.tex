\subsection{Hierarchical parameterization}

An FWI-suitable parameterization of orthorhombic media should describe isotropic media using only three non-zero parameters -- $V_p,V_s,$ and $\rho$. 

By adding three more parameters -- $\varepsilon_1, \eta_1,$ and $\gamma_1$ -- we can describe VTI media.

\todo{
Parameterization suggested by all and Alkhalifah features this separation of parameters into three groups. First group specifies parameters in isotropic medium. This parameters are standard and include PV velocity and S vein velocity and density. Second group brings in vertically transversely isotropic parameters. This group includes widespread in exploration and seismic parameterization by Alkhalifah and 20. Namely it includes Epsilon, gamma and eta. In this parameterization Epsilon describes the elliptic part of of anisotropy of P waves. Eta either corrects the anisotropy of P waves completely defeat their kinetics kinematics. The gamma parameter describes the anisotropy of SH waves. SV waves are thematically sensitive to the same parameters as P waves.
}







Finally, four additional parameters -- $\varepsilon_d, \eta_d, \gamma_d$, and $\delta_3$ -- allow us to go from VTI to orthorhombic anisotropy. This kind of parameterization is presented in \cite{juwon2016}.


Radiation patterns and the spectral sensitivities for any parameterization can be obtained through variable substitution to $C_{ij},\rho$ parameterization. 
\beq \label{eq:JacoCij}
\bm{\Rp}_{\mv, WT} = \bm{\Rp}_{\mv_0(\mv), WT} = \bm{\Rp}_{\pd{\mv_0}{\mv}\mv}
\eeq

The Jacobian $\pd{(\nmz{C}_{ij},\nmz{\rho})}{\mv}$ for the hierarchical parameterization has the following form:
\beq \label{eq:Jaco}
\delta \left(\begin{array}{c}
	 \nmz{\rho}  \\
	\nmz{C}_{11} \\
	\nmz{C}_{22} \\
	\nmz{C}_{33} \\
	\nmz{C}_{12} \\
	\nmz{C}_{13} \\
	\nmz{C}_{23} \\
	\nmz{C}_{44} \\
	\nmz{C}_{55} \\
	\nmz{C}_{66}
\end{array}\right)
= \left(
\begin {array}{cccccccccc} 
1 & 0 & 0 & 0 & 0 & 0 & 0 & 0 & 0 & 0 \\ 
1 & 2 & 0 & 0 & 0 & 0 & 0 & 0 & 0 & 0 \\ 
1 & 2 & 0 & 0 & 2 & 0 & 0 & 0 & 0 & 0 \\
1 & 2 & 0 & -2 & 0 & 0 & 0 & 0 & 0 & 0 \\ 
1-2\varkappa & 2 & -4\varkappa & 0 & 0 & 0 & 0 & 1 & -4\varkappa & 0 \\
1-2\varkappa & 2 & -4\varkappa & -1 & 0 & -1 & 0 & 0 & 0 & 0 \\
1-2\varkappa & 2 & -4\varkappa & -1 & 1 & -1 & -1 & 0 & 0 & -4\varkappa \\
\varkappa & 0 & 2\varkappa & 0 & 0 & 0 & 0 & 0 & 0 & 2\varkappa\\
\varkappa & 0 & 2\varkappa & 0 & 0 & 0 & 0 & 0 & 0 & 0 \\
\varkappa & 0 & 2\varkappa & 0 & 0 & 0 & 0 & 0 & 2\varkappa & 0
\end {array}
 \right)
\delta \left(\begin{array}{c}
	  \nmz{\rho}     \\
	 \nmz{V}_{p}     \\
	 \nmz{V}_{s}     \\
	 \eps_1    \\
	 \eps_d    \\
	 \eta_1    \\
	 \eta_d    \\
	\delta_3   \\
	\gamma_1   \\
	\gamma_d 
\end{array}\right)
\eeq
%
We also present the partial derivatives of the $C_{ij}$ parameters in a graph (\figref{PP_Full/parDerivCij}) to illustrate the perturbations in the elastic parameters that are introduced by perturbations from a single hierarchical parameter. To simplify formulas and focus on geometric properties of scattering we normalize units in the following way -- velocities are normalized by the background $\v_{p0}$, densities by $\rho_0$, elastic constants by $\rho_0 \v_p^2$. The normalized quantities are distinguished by the dash, e.g.
\beq
\nmz{\v}_p \equiv \frac{\v_p}{\v_{p0}}, 
\nmz{\v}_s \equiv \frac{\v_s}{\v_{p0}},
\nmz{C}_{55} \equiv \frac{C_{55}}{\rho\v^2_{p0}}. 
\eeq  

\plot{PP_Full/parDerivCij}{width=\columnwidth}{Partial derivatives of $C_{ij}$ parameters w.r.t. hierarchical parameterization parameters for $V_{0s}/V_{0p}=\varkappa=\frac{1}{\sqrt{3}}$.}


An advanced choice of parameterization will give some insights that are not straightforward from the $C_{ij}$ parameterization. For example, parameters contributing to $V_p$ provide zero sensitivity to $S$-wave scattering. Being a combination of nine $C_{ij}$ parameters, $V_p$ is hardly discoverable in the original parameterization. Thus, our knowledge is still limited by the choice of parameterization, as we cannot visually identify coupling among three or more parameters.
