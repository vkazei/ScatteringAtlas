\subsection{Hierarchical parameterization}

In order to obtain more insight into scattering from orthorhombic parameters and look at some scattering features for VTI and isotropic parameters we use parameterization introduced by \cite{juwon2016}. The parameterization splits ten orthorhombic parameters into three groups: isotropic, VTI, orthorhombic, so that if VTI and orthorhombic parameters are zero, the medium is isotropic and if orthorhombic parameters are zero, then the medium is VTI.  
Isotropic media is described by three non-zero parameters -- $V_p,V_s,$ and $\rho$. %These are standard and include velocities of P and S waves and density.

By adding three more parameters -- $\varepsilon_1, \eta_1,$ and $\gamma_1$ -- we describe VTI media. This specific choice of parametrization for VTI reduces the trade offs for surface seismic data inversion objectives as shown by \cite{alkhalifah2014}. In this parameterization $\epsilon$ describes the elliptic part of of anisotropy of $P$ waves. Parameter $\eta$ represents the anelliptic part of the anisotropy of $P$ waves. The $\gamma$ parameter describes the anisotropy of horizontally polarized $SH$ waves. In VTI medium $SH$ waves show elliptic anisotropy and therefore one parameter is sufficient to describe their kinematics. Hence we work with orthorhombic media we add $1$ indexes to reflect that this parameters belong to the $x_1-x_3$ plane and the other vertical plane of symmetry $x_2-x_3$ may have different VTI behavior. 


Finally, four additional parameters -- $\varepsilon_d, \eta_d, \gamma_d$, and $\delta_3$ -- allow us to go from VTI to orthorhombic anisotropy. As we mentioned before there are two vertical planes of symmetry in orthorhombic medium. Wave propagation in the first plane $x_1-x_3$ is fully defined by the above mentioned parameters, the plane $x_2-x_3$, on the other hand, can have different parameters. Say those are $\varepsilon_2, \eta_2,$ and $\gamma_2$, we can introduce the deviation parameters 
\beq
\varepsilon_d = \frac{\varepsilon_2 - \varepsilon_1}{1+2\varepsilon_1},
\\
\eta_d = \frac{\eta_2 - \eta_1}{1+2\eta_1},
\\ 
\gamma_d = \frac{\gamma_2 - \gamma_1}{1+2\gamma_2},
\eeq
that will tell us how different are the two vertical planes of symmetry of orthorhombic medium. Additional parameter $\delta_3$ determines the anellipticity of $P$ wave anisotropy in the $x_1-x_2$ plane \citep{juwon2016}. Explicit equations that can be used to derive the elastic stiffness matrix elements from the hierarchical parameters can be found in the Glossary. 

Radiation patterns and the spectral sensitivities for any parameterization can be obtained through variable substitution to $C_{ij},\rho$ parameterization. 
\beq \label{eq:JacoCij}
\bm{\Rp}_{\mv, WT} = \bm{\Rp}_{\mv_0(\mv), WT} = \bm{\Rp}_{\pd{\mv_0}{\mv}\mv}
\eeq

The Jacobian $\pd{(\nmz{C}_{ij},\nmz{\rho})}{\mv}$ for the hierarchical parameterization has the following form:
\beq \label{eq:Jaco}
\delta \left(\begin{array}{c}
	 \nmz{\rho}  \\
	\nmz{C}_{11} \\
	\nmz{C}_{22} \\
	\nmz{C}_{33} \\
	\nmz{C}_{12} \\
	\nmz{C}_{13} \\
	\nmz{C}_{23} \\
	\nmz{C}_{44} \\
	\nmz{C}_{55} \\
	\nmz{C}_{66}
\end{array}\right)
= \left(
\begin {array}{cccccccccc} 
1 & 0 & 0 & 0 & 0 & 0 & 0 & 0 & 0 & 0 \\ 
1 & 2 & 0 & 0 & 0 & 0 & 0 & 0 & 0 & 0 \\ 
1 & 2 & 0 & 0 & 2 & 0 & 0 & 0 & 0 & 0 \\
1 & 2 & 0 & -2 & 0 & 0 & 0 & 0 & 0 & 0 \\ 
1-2\varkappa & 2 & -4\varkappa & 0 & 0 & 0 & 0 & 1 & -4\varkappa & 0 \\
1-2\varkappa & 2 & -4\varkappa & -1 & 0 & -1 & 0 & 0 & 0 & 0 \\
1-2\varkappa & 2 & -4\varkappa & -1 & 1 & -1 & -1 & 0 & 0 & -4\varkappa \\
\varkappa & 0 & 2\varkappa & 0 & 0 & 0 & 0 & 0 & 0 & 2\varkappa\\
\varkappa & 0 & 2\varkappa & 0 & 0 & 0 & 0 & 0 & 0 & 0 \\
\varkappa & 0 & 2\varkappa & 0 & 0 & 0 & 0 & 0 & 2\varkappa & 0
\end {array}
 \right)
\delta \left(\begin{array}{c}
	  \nmz{\rho}     \\
	 \nmz{V}_{p}     \\
	 \nmz{V}_{s}     \\
	 \eps_1    \\
	 \eps_d    \\
	 \eta_1    \\
	 \eta_d    \\
	\delta_3   \\
	\gamma_1   \\
	\gamma_d 
\end{array}\right)
\eeq
%
We also present the partial derivatives of the $C_{ij}$ parameters in a graph (\figref{PP_Full/parDerivCij}) to illustrate the perturbations in the elastic parameters that are introduced by perturbations from a single hierarchical parameter. For example, perturbing $V_p$ is equivalent to simultaneously perturbing all $C_{ij}$ with $i \leq 3$ and $j \leq 3$ (see the second column in \eqrf{Jaco}); and perturbing $\gamma_1$ is equivalent to simultaneously perturbing $C_{12}$ and $C_{66}$ (see the ninth column in the matrix in \eqrf{Jaco}). To simplify formulas and focus on geometric properties of scattering we normalize units in the following way -- velocities are normalized by the background $\v_{p0}$, densities by $\rho_0$, elastic constants by $\rho_0 \v_p^2$. The normalized quantities are distinguished by the dash, e.g.
\beq
\nmz{\v}_p \equiv \frac{\v_p}{\v_{p0}}, 
\nmz{\v}_s \equiv \frac{\v_s}{\v_{p0}},
\nmz{C}_{55} \equiv \frac{C_{55}}{\rho\v^2_{p0}}. 
\eeq  

\plot{PP_Full/parDerivCij}{width=\columnwidth}{Partial derivatives of $C_{ij}$ parameters w.r.t. hierarchical parameterization parameters. Throughout the paper, the Poisson ratio is assumed to be equal to 0.25 ($V_{0s}/V_{0p}=\varkappa=\frac{1}{\sqrt{3}}$), which in fact affects only the derivatives w.r.t. density.}


An advanced choice of parameterization will give some insights that are not straightforward from the $C_{ij}$ parameterization. For example, parameters contributing to $V_p$ provide zero sensitivity to $S$-wave scattering. Being a combination of nine $C_{ij}$ parameters, $V_p$ is hardly discoverable in the original parameterization. Thus, our knowledge is still limited by the choice of parameterization, as we cannot visually identify coupling among three or more parameters.
