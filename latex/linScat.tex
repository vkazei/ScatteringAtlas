\section{Wavenumber illumination theory}
In the previous section, we discussed the scattering of plane waves on point 
scatterers. But reflection-based radiation patterns can be 
remapped into the frequency-wavenumber domain in order to accommodate the scattering effects of any arbitrary perturbations that fit into the Born approximation.
%
\cite{ewald1969} spheres have been used to recognize the relation between illuminated wavenumbers in a perturbation and the directions of its incident and scattered waves in monochromatic plane-wave experiments. 
For example, \cite{devaney1984} used this 
method to quantify the resolution of geophysical diffraction tomography images in the
wavenumber domain. This technique was later applied to acoustic FWI scenarios 
for surface seismic acquisition \citep{mora1989} and vertical seismic profiling 
\citep{wu:11}. \cite{kazei2013gp} introduced the amplitude of the spectral 
sensitivity of the data to the perturbations of various background media, and applied it to models with diving waves and multiples \citep{kazei2015seg,kazei2013spectral}. An elastic isotropic-background case 
with VTI perturbations was considered by \cite{podgornova}. Here, we show the 
applicability of this technique to arbitrary anisotropic perturbations of isotropic media. 
%
% 
%The possibility of recovering a set of parameters is defined by the sensitivity 
%of the scattered wavefield to these parameters. Under the plane-wave 
%assumption, the resolution is straightforward for the wavenumber domain 
%representation of perturbation.
%For single wave mode we could compare reflection-based radiation patterns to 
%understand the coupling effects.


In this section, we provide a derivation of the 
scattering function and 
generalize it to spectral sensitivities 
\citep{kazei2013gp,podgornova,kazei2017}.
%
By incorporating the spectral sensitivities, we can consider a broad class of anomalies 
so small that we can neglect the second-order scattering. 
%
 

%For instance, mapping into wavenumber domain allows to consider 
%information 
%available for inversion from $P-P$ and $P-SV$ waves together.
% 
\subsection{Far-field plane-wave scattering}
Radiation patterns simply represent different types of scattered wavefields for a point scatterer 
when the far-field or plane-wave approximation is applied to \eqrf{hudson}. While the derivation is well-known, we return to it in order to generalize the radiation patterns to arbitrary scatterers that 
satisfy \eqrf{hudson}. First, we utilize the approximation from the receiver 
side of \eqrf{hudson}, since it includes Green's tensors and not incident 
wavefields. Then, we treat the source wavefields in the same manner.

\subsubsection{The Born approximation}
We consider an incident wavefield $\Uv^0$ on a perturbation of the density ($\delta\rho(\xv)$) and stiffness tensor ($\delta c_{kjpq}(\xv)$) parameters.
The frequency-domain Born approximation for the scattered wavefield $\delta \Uv$ in an anisotropic elastic medium is \citep[e.g.,][]{hudson1981,beylkin1990,shaw2004}
\beq \label{eq:hudson}
\delta U_i (\xv_s,\xv_\gv,\omega) = \intyV  (\delta \rho(\xv) U^0_k \omega^2 - 
\delta c_{kjpq}(\xv) \pd{U^0_p}{x_q}  \pd{}{x_j})G_{ik}(\xv_g, \xv) \d \xv,
\eeq
where $\omega$ is the angular frequency, $\xv_s$ and $\xv_g$ are the coordinates of the source and receiver, respectively, and $G_{ik}(\xv_g, \xv)$ is the Green's tensor for the background medium. Integration is performed over the volume of the perturbation $\delta \mv = (\delta \rho, \delta c_{ijpq})$, where $\mv$ is the vector of unperturbed parameters. 
We can vectorize equation~\eqref{eq:hudson},
\beq \label{eq:hudsonVec}
\delta \Uv (\xv_s,\xv_\gv,\omega) = \intyV  (\omega^2 \delta \rho(\xv) \Uv^0 (\xv) \cdot\Gv (\xv) - 
\nabla \Gv (\xv) : \delta \cv(\xv) : (\nabla \Uv^0 (\xv))   \d \xv,
\eeq
which already gives us some insight into the scattering. For example, scattering on the density perturbation is determined by the kernel 
\beq
\omega^2 \delta \rho(\xv) \Uv^0 (\xv) \cdot\Gv (\xv),
\eeq 
and, hence, depends only on the amplitudes and polarization of the incident $\Uv^0$ and scattered $\Gv$ wavefields. For instance, if we consider the scattering of $SH$ waves in a vertical plane from a perturbation of density, then the amplitude of the scattered wavefield will be constant.
%\figref{SHSH/SHSHCij}. 
At the same time, the scattering from perturbations of elastic constants $C_{ij}$, which are driven by the kernel
\beq
\nabla \Gv (\xv) : \delta \cv(\xv) : (\nabla \Uv^0 (\xv)),
\eeq
can also depend on the propagation directions
of the incident and scattered wavefields, which are extracted by the gradient $\nabla$ operator.
 
\subsubsection{Wavefields scattered by perturbation}
The Green's tensor $G_{ik}$ in the Born approximation~\eqref{eq:hudsonVec} physically 
takes care of the propagation of a scattered wavefield from the perturbation to the 
receiver.
The far-field approximation for the Green's tensor \citep[e.g.,][]{snieder2002} in a homogeneous, isotropic background 
%is
%\begin{align} \label{eq:farField}
%	\Gv(\xv,\xv_g) &= \alpha\bigg(  e^{i\frac{\omega}{\v_p}||\xv - \xv_g||} \gv\gv
%	\\
%	&+\frac{1}{\varkappa^2} e^{i\frac{\omega}{\v_s}||\xv - \xv_g||} (\gv_{\theta}\gv_{\theta} + \gv_{\phi}\gv_{\phi})\bigg), 
%	\varkappa = \frac{\v_s}{\v_p}, \alpha = \frac{1}{4\pi\rho \v_p^2||\xv - \xv_g||},
%\end{align}
%where $\gv,~\gv_{\theta},$ and $~\gv_{\phi}$ are the polarization vectors of the $P$, $SV$, and $SH$ waves, respectively. Thus, the far-field part of the Green's tensor (26) 
consists of $P,~ SV,$ and $SH$ waves:
\beq \label{eq:3G}
\Gv(\xv_g, \xv) = \alpha (\Gv_P(\xv_g, \xv) + \Gv_{SV}(\xv_g, \xv) + \Gv_{SH}(\xv_g, \xv)). 
\eeq
While $SV$ and $SH$ waves in isotropic media most often come together, they are usually recorded separately and therefore can be analyzed separately too. 
The coefficient 
\beq
\alpha = \frac{1}{4\pi\rho \v_p^2||\xv - \xv_g||}.
\eeq
is dropped in our resolution analysis, since it is related to geometrical spreading and can be effectively compensated using pseudo-hessian techniques. Different components of the Green's function are approximated locally by plane waves. For the first component $\Gv_P$, the approximation leads to
\beq \label{eq:GvP}
\Gv_P = e^{i\frac{\omega}{\v_p}||\xv_g- \xv ||} \gvn\gvn \simeq 
e^{i\frac{\omega}{\v_p}\gv\cdot(\xv - \xv_g)}\gvn\gvn, ~ 
\gvn = \frac{\gv}{|\gv|} = \frac{\xv-\xv_g}{|\xv-\xv_g|},
\eeq
which is the $P$ wave approximated by a plane wave at the location $\xv$. ($\gv\gv)_{ik} = g_i g_k$ denotes standard outer product of vector $\gv$ with itself. 
%For the second and third 
%components, $\Gv_{SV}$ and $\Gv_{SH}$, similar expressions can be provided 
%\citep{snieder2002}.
The second component, $\Gv_{SV}$, represents a shear wave polarized in the vertical plane 
that contains $\xv$ and $\xv_g$:
\beq \label{eq:GvSV}
\Gv_{SV} = \frac{1}{\varkappa^2} e^{i\frac{\omega}{\v_s}||\xv - \xv_g||}\gv_{\theta}\gv_{\theta} \simeq 
\frac{1}{\varkappa^2} e^{i\frac{\omega}{\v_s}\gv\cdot(\xv - \xv_g)}\gv_{\theta}\gv_{\theta},~
\gv_\theta = \left( \frac{\gv \times \ev_z}{|\gv \times \ev_z|}\times \gvn \right). 
\eeq
Finally, the third component of the Green's tensor, $\Gv_{SH}$, is an $SH$ wave polarized along $\gv_\phi$:
\beq \label{eq:GvSH}
\Gv_{SH} = \frac{1}{\varkappa^2} e^{i\frac{\omega}{\v_s}||\xv - \xv_g||}\gv_{\phi}\gv_{\phi} \simeq 
\frac{1}{\varkappa^2} e^{i\frac{\omega}{\v_s}\gv\cdot(\xv - \xv_g)}\gv_{\phi}\gv_{\phi},
\gv_\phi = \left( \frac{\gv \times \ev_z}{|\gv \times \ev_z|}\right),
\eeq
which is the horizontal component of a shear wave.
%\subsubsection{Incident wavefield}

For the incident wavefield we consider point forces in three directions $\svn, \sv_\phi, \sv_\theta$, leading to three respective wavetypes:
\beq \label{eq:P_sou}
\Uv_{0P} = \Gv_P(\xv_s,\xv) \cdot \svn,
\eeq
\beq
\Uv_{0SV} = \Gv_{SV}(\xv_s,\xv) \cdot \sv_\theta,
\eeq
\beq
\Uv_{0SH} = \Gv_{SH}(\xv_s,\xv) \cdot \sv_\phi.
\eeq
%\cite{walter2007} analyzed the empirical relations between the amplitudes of $P$ 
%and $S$ waves emitted by explosive sources and double-couple earthquake sources, and came to the conclusion that $P$ waves dominate the explosions, while $S$ waves dominate the earthquakes.

%For simplicity, we consider incident plane waves with unit amplitudes:
%\beq
%\Uv_0 = \Av_0 \exp(i \kv_s \cdot \xv),
%\eeq
%where $\Av_0$ and $\kv_s$ represent the polarization and wavenumber vectors, respectively, of an incident wave.
%For $P$ waves,
%\beq \label{eq:P_sou}
%\Av_0 = \svn,~ \kv_s=\frac{\omega}{\v_p}\svn, \svn = \frac{\xv_s-\xv}{|\xv-\xv_s|}.
%\eeq
%For $SV$ waves,
%\beq
%\Av_0 = \sv_\theta,~ \kv_s=\frac{\omega}{\v_s}\svn,~
%\sv_\theta =  \left( \frac{\sv \times \ev_z}{|\sv \times \ev_z|}\times \svn
%\right).
%\eeq
%For $SH$ waves,
%\beq
%\Av_0 = \sv_\phi,~ \kv_s=\frac{\omega}{\v_s}\svn,~
%\sv_\phi = \left( \frac{\sv \times \ev_z}{|\sv \times \ev_z|}\right).
%\eeq


%
%\input{theory/born}
%
\subsection{Model wavenumbers and scattering modes}
We start with scattered $P$ waves as they are the easiest to handle and by far the most popular in FWI applications. By substituting equations \eqref{eq:P_sou} and \eqref{eq:GvP} into equation \eqref{eq:hudson}, and projecting the scattered wavefield $\Uv$ onto the direction of $\gv$ (from the perturbation to the receiver), we obtain the approximate amplitude of the scattered $P-P$ waves \citep{kazei2018} as
\beq \label{eq:UPPInt}\label{eq:UPPInt1}
\delta U_{PP} = \delta \Uv_P \cdot \svn  \propto 
%\frac{\delta \mathbf{U} \cdot \gv}{\alpha_{PP}} \simeq
\intyV   e^{i\Kv \cdot \xv} (\svn\cdot\gvn \delta \nmz{\rho} (\xv) + \svn\svn : \delta 
\nmz{\cv}(\xv) :\gvn\gvn) \d \xv, 
\eeq
where
%\beq 
%\alpha_{PP} = \frac{\omega^2}{v^2_p} \alpha_s \alpha_g e^{i\frac{\omega}{\v_p}(-\xv_s\sv-\xv_g\gv)}
%\eeq
%and
\beq \label{eq:K}
\Kv = \omega(\sv+\gv).
\eeq

Analogously, for the amplitude of the scattered $P-SV$ waves, we use equations \eqref{eq:P_sou} and \eqref{eq:GvSV}, and project \eqref{eq:hudson} onto vector $\gv_{\theta}$ to obtain
\beq \label{eq:UPSVInt}
\delta U_{PSV} = \delta \Uv_P \cdot \sv_\theta 
\propto   
%\frac{\delta \mathbf{U} \cdot \gv_\theta}{\alpha_{PS}} \simeq\frac{1}{\varkappa^3}
\intyV  e^{i\Kv_{PS}\cdot\xv} (\svn\cdot\gv_{\theta} \delta \nmz{\rho} + \frac{1}{\varkappa} \svn\svn : \delta \nmz{\cv} :\gvn\gv_{\theta}) \d \xv. 
\eeq
%where
%\beq
%\alpha_{PS} = \frac{\alpha_{PP}}{\varkappa^3} e^{i\omega\gv_\theta \cdot (\frac{\xv_g}{\v_p}-\frac{\xv_g}{\v_s})}, \Kv_{PS}=\frac{\omega}{\v_p}\sv+\frac{\omega}{\v_s}\gv.
%\eeq

Replacing the polarization vector $\gv_\theta$ in (39) with $\gv_\phi$ is sufficient to obtain similar expressions for the $P-SH$ scattering scenario. For $S$ waves, we obtain similar expressions by using equations \eqref{eq:P_sou}, \eqref{eq:GvSH}, and \eqref{eq:hudson}:
\beq \label{eq:USHSHInt}
\delta U_{SHSH} = \delta \Uv_{SH} \cdot {\sv_\phi} 
\propto 
%\equiv  \frac{\delta \mathbf{U} \cdot \gv_\phi}{\alpha_{SS}} \simeq \frac{1}{\varkappa^3}
\intyV  e^{i\Kv_{SS}\cdot\xv} (\sv_\phi \cdot \gv_\phi \delta \rho + \frac{1}{\varkappa^2} \nmz{\sv}\sv_\phi : \delta \nmz{\cv} :\nmz{\gv}\gv_{\phi}) \d \xv. 
\eeq
%where
%\beq
%\alpha_{SS} = \frac{\alpha_{PP}}{\varkappa^6} e^{i\omega\gv \cdot (\frac{\xv_g}{\v_s}-\frac{\xv_g}{\v_s})}, 
%\Kv_{SS}=\frac{\omega}{\v_s}(\sv+\gv).
%\eeq

For different types of scattering, our further resolution analysis is based on equations 
(\ref{eq:UPPInt1}-\ref{eq:USHSHInt}) and their analogs, which describe the monochromatic scattering of $P$ waves.

 

Different scattering modes contribute to different wavenumbers at 
the same reflection angle and frequency of the image. Therefore, in order to investigate all the
information available, we need to remap the reflection-based radiation patterns into 
the spatial wavenumber domain. Equation \eqref{eq:K} governs the mapping procedure and \figref{K_PP_Full} illustrates how it defines the resolution of the different types of body wave scattering.

\tplot{K_PP}{K_PS}{K_SS}{For the same incident angle 
	and same temporal frequency different modes provide illumination to different 
	vertical wavenumbers. (a) $P-P$ waves illuminate the lowest wavenumbers (good for FWI), 
	(c) $S-S$ waves -- the highest wavenumbers (good for migration), and (b) $P-S$ waves -- the intermediate wavenumbers.}

We note that the right-hand sides of equations \eqref{eq:UPPInt1}, \eqref{eq:UPSVInt}, and \eqref{eq:USHSHInt} are essentially Fourier transforms of the perturbation. Thus, we can generalize these equations as 
\beq \label{eq:KvMain}
\delta U_{WI-WS}(\sv,\gv,\omega) \propto \spv\cdot\gpv \delta \hat{\rho}(\Kv_{WI-WS}(\sv,\gv,\omega)) + \sv\spv : \delta \hat{\cv}(\Kv_{WI-WS}(\sv,\gv,\omega)) :\gv\gpv.
\eeq
Equation~\eqref{eq:KvMain} shows that for a given scattering mode, the wavenumber resolved by inversion depends only on the frequency, source and receiver directions. On the other hand, we can infer all possible directions towards the source and receiver for a given wavenumber. Analyzing scattering for these directions leads directly to irresolvable tradeoffs in inversion at a particular length.

%In single mode scattering considerations, the coefficient $\varkappa^{-3}$ can 
%be dropped \citep{podgornova}. 

%but in order to consider the information 
%available from several wave types together we need to take it into account:
%\beq \label{eq:alphaPPPS}
%|\alpha_{PS}| = \varkappa^{-3}|\alpha_{PP}|
%\eeq
%\beq
%\label{eq:incident}
%\kv_g = \frac{\omega}{v_{WS}}\gv,  \kv_s = \frac{\omega}{v_{WI}}\sv.
%\eeq


%


