\subsection{Scattering function}
A monochromatic plane wave is defined by its phase, polarization, and 
propagation directions. The phase accumulated while traveling through 
a background medium is dropped in our analysis so that we can focus on the local scattering 
features. Therefore, here the scattered wavefield $\delta 
\Uv$ depends only on the incident plane-wave slowness vector $\sv$, the scattered plane-wave slowness vector $\gv$, the respective 
polarizations $\spv$ and $\gpv$, the frequency $\omega$, and the 
perturbation of the parameters 
$\delta \mv  = (\delta \rho(\xv), \delta \cv(\xv))^T$:
\beq
\delta \Uv \cdot \gpv  \propto \SF_{\delta \mv} (\sv,\gv, \spv, \gpv).
\eeq
%
The coefficient $\SF$, well known for its arbitrary point scatterers, is called the scattering function \citep{eaton1994, shaw2004, 
calvet2006, kazei2018}, and is defined as 
%
\beq \label{eq:sfun}
\SF_{\delta \mv} (\sv,\gv, \spv, \gpv) \equiv \sp_k \gp_k \delta \rho  + 
s_i \sp_j \delta c_{ijkl} g_k \gp_l 
\\
= \spv \cdot \gpv \delta \rho + \sv \tmul \spv : \delta \cv : \gv \tmul \gpv.
\eeq
While $\SF$ is \emph{independent} of the background medium, which can have up to triclinic-anisotropy complexity, eight 
different angles determine the directions of polarization and propagation for its incident and scattered wavefield (there are four directions, featuring two angles each). 
Thus, there are many possibilities for polarization and propagation relations in a general anisotropic background as well as for velocities of incident and scattered waves; here, we consider scattering in an isotropic background.
 
%Instead of expressions including angles we try to stick with vector notation to keep it more compact.