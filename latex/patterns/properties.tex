\section{Analytic properties of scattering radiation patterns}
In this section, we discuss the analytic properties of reflection-based patterns. These include reciprocity, reduction, symmetry, relation to the linearized reflection coefficient, and transmission patterns.
\subsection{Reciprocity}
Symmetry of the elastic tensor $c_{ijkl}$ leads to the reciprocity 
for scattering functions:
\beq
\SF_{\delta\mv} (\sv,\spv,\gv,\gpv) = 
\spv \cdot \gpv \delta \rho + \sv \tmul \spv : \delta \cv : \gv \tmul \gpv = \\
\gpv \cdot \spv \delta \rho +  \gv \tmul \gpv : \delta \cv : \sv \tmul \spv = 
\SF_{\delta\mv} (\gv,\gpv,\sv,\spv). 
\eeq
and diffraction-based radiation patterns:
\beq
\Dp_{WI-WS,\delta\mv} (\sv,\gv) = \Dp_{WS-WI,\delta\mv} (\gv,\sv).
\eeq 
This property is obvious, yet it can sometimes help to debug radiation patterns. 

\subsection{Reduction for isotropic parameters}
For isotropic parameters ($\lambda, \mu, \rho$), the \dP depend only on the 
opening scattering angle, defined by the scalar product $\sv\cdot\gv$ \citep[e.g.,][]{wu1985}: 
\beq
\Dp_{iso,WI-WS} (\sv,\gv) = \Dp_{iso,WI-WS} (\sv \cdot \gv).
\eeq
Therefore, they are easily represented graphically as functions of a single scalar. 

\subsection{Reduction for TI parameters}
For transversely isotropic parameters, \dP are invariant w.r.t. rotation around 
the symmetry axis $\ev^{TI}$. Thus, the number of angles that 
define scattering can be reduced to three \citep{calvet2006}:
\beq
\Dp_{TI,WI-WS} (\sv,\gv) = \Dp_{TI,WI-WS} (\sv \cdot \gv, \sv \cdot \ev^{TI}, \gv 
\cdot \ev^{TI}).
\eeq
Here, $TI$ denotes any parameter of the tilted transversely isotropic medium, and $\sv \cdot \ev^{TI}$ and $\gv 
\cdot \ev^{TI}$ give the incident and scattered angles, respectively, measured from the symmetry axis $\ev^{TI}$ of the background medium.

\subsection{Symmetry}
A radiation pattern is antisymmetric w.r.t. $\sv$ for density $\rho$, i.e., 
\beq \label{eq:sym_rho}
\Dp_{\rho, WI-WS} (-\sv,\gv) = -\Dp_{\rho,\sdot} (\sv \cdot \gv),
\eeq
and symmetric for all elastic constants $c_{ijkl}$:
\beq \label{eq:sym_Cij}
\Dp_{\cv, WI-WS} (-\sv,\gv) = \Dp_{\cv,\sdot} (\sv \cdot \gv).
\eeq
The consequence of (15) is that the density in full diffraction-based radiation patterns can always be distinguished from scattering by other parameters. Due to reciprocity, the same symmetry properties hold for the vector $\gv$. In other than $C_{ij}, \rho$ parameterizations which include density \citep[e.g.]{tsvankin1997,stovas2015,juwon2016}, the symmetry properties \eqref{eq:sym_rho} may not apply.
 
\subsection{Relation to linearized reflection coefficients}
Reflection-based radiation patterns are proportional to the linearized reflection coefficients 
\citep{shaw2004}, even though patterns are derived from Born approximation and reflection coefficients are derived from boundary conditions at a solid-solid interface. However, unlike the reflection-based radiation patterns, the coefficients depend on the scattering type and not the perturbed parameters.

\subsection{Transmission-reflection pattern relation}
For monotypic (WI = WS) waves, the so-called transmission radiation patterns can be defined as
\beq
\Tp_{WI,\delta \mv} (\gv) \equiv \Dp_{WI-WI,\delta \mv} (-\gv, \gv) = \pd{c_{ijkl}}{c_{pqrs}} g_i \gp_j g_k \gp_l. 
\eeq

One interesting property of the transmission-based radiation patterns in $C_{ij},\rho$ is that, for all monoclinic elastic parameters except density, they do not differ in shape from the reflection-based radiation patterns. The proof of this relation we provide below.

We first rewrite the reflection-based radiation pattern expression:
\beq
\Rp_{c_{pqrs},WI-WI} (\gv) \equiv \pd{c_{ijkl}}{c_{pqrs}} 
(-1)^{1+\delta_{i3}} g_i  (-1)^{1+\delta_{j3}} \gp_j g_k \gp_l = 
%(-1)^{\delta_{i3}+\delta_{j3}} \pd{c_{ijkl}}{c_{pqrs}} g_i \gp_j g_k \gp_l =
(-1)^{N(i,j)} \pd{c_{ijkl}}{c_{pqrs}} g_i \gp_j g_k \gp_l,
\eeq
where we introduce a function 
\beq
\nt (p,q) \equiv  \delta_{p3}+\delta_{q3}
\eeq
that counts the number of elements equal to three in a sequence of numbers.


For monoclinic parameters $c_{pqrs}$ $\nt(p,q)$ and $\nt(r,s)$ are either odd or even together
\beq
\nt (p,q) \equiv \nt(r,s) \Mod 2, 
\eeq
which is required by the symmetry of elastic equations w.r.t. the change in the direction of the third axis.  

If $\nt (p,q)$ is even (either equal two or zero)
$$
\nt (p,q) \equiv \nt(r,s) \equiv 0 \Mod 2
$$ 
i.e., for scattering on the orthorhombic parameters $\Cv_{ij},$ where $i \leq 3$ and $j \leq 3$,  and $\Cv_{66}$, and the monoclinic parameters $\Cv_{16},\Cv_{26}$, and $\Cv_{36}$ (in Voigt notation), then
\beq \label{eq:transEqRefl}
\Rp_{c_{pqrs},WI-WI} (\gv) \equiv \pd{c_{ijkl}}{c_{pqrs}} g_i \gp_j g_k \gp_l = \Tp_{c_{pqrs},WI} (\gv).
\eeq
On the other hand, if $\nt (p,q) \bmod 2 = \nt(r,s) \bmod 2 = 1$, i.e., for scattering on $\Cv_{44},\Cv_{55}$, and the monoclinic parameter $\Cv_{45}$, then
\beq \label{eq:transEqMRefl}
\Rp_{c_{pqrs},WI-WI} (\gv) \equiv -\pd{c_{ijkl}}{c_{pqrs}} g_i \gp_j g_k \gp_l = -\Tp_{c_{pqrs},WI} (\gv).
\eeq

%\begin{figure}
%	\centering
%	\includegraphics[width=0.7\linewidth]{CijWiki}
%	\caption{}
%	\label{fig:CijWiki}
%\end{figure}




\subsection{Monoclinic -- monotypic}

For non-monoclinic parameters $c_{pqrs}$,
\beq
N(p,q) \neq N(r,s) \bmod 2.
\eeq
Without any loss of generality we assume that $N(p,q) mod 2 = 1$
and in particular $p=3$. Automatically $q\neq p$ and therefore:
\beq
\Rp_{c_{3qrs},WI-WI} (\gv) \equiv
%
(-1)^{\delta_{i3}+\delta_{j3}} \pd{c_{ijkl}}{c_{3qrs}} g_i \gp_j g_k \gp_l = \\
%
- \pd{c_{3qkl}}{c_{3qrs}} g_p \gp_q g_k \gp_l 
- \pd{c_{q3kl}}{c_{3qrs}} g_q \gp_p g_k \gp_l 
+ \\
\pd{c_{kl3q}}{c_{3qrs}} g_p \gp_q g_k \gp_l
+ \pd{c_{klq3}}{c_{3qrs}} g_q \gp_p g_k \gp_l \equiv 0.
\eeq

We link the vertical resolution directly to the reflection-based radiation patterns in the next section, which finalizes our review of the radiation-pattern properties. 

\section{Born approximation validity} 
The Born approximation for scattering (\eqrf{hudson}) is valid for relatively small
\beq \label{eq:assContr}
\frac{|\delta \mv|}{|\mv|} << 1,
\eeq
and local
\beq \label{eq:assLocal}
\frac{\omega d}{V_0}\frac{|\delta \mv|}{|\mv|} << 1,
\eeq
perturbations $\delta \mv$, where $V_0$ is the smallest background velocity ($V_s$ in the isotropic case), and $d$ is the diameter of the perturbation. The above assumptions are best met in the case of low-contrast point scatterers. For instance, the approximation is valid for perturbations in thin, low-contrast layers and near-normal wave incidence angles \citep{shaw2004}. %

The Born approximation does not imply specific shape restrictions on the scattering 
perturbation or the background medium. Therefore, modern methods for the dynamic 
inversion of seismic wavefields, including linearized reverse-time migration 
(LSRTM) and every iteration of full-waveform inversion (FWI), rely heavily on it. 
Thus, we can analyze the resolution of these methods by exploring the 
sensitivity kernels in the spatial Fourier domain \citep[e.g.,][]{devaney1984,wu:11,mora1989,sirgue2004,kazei2013gp,kazei2013spectral,podgornova,kazei2017}.

%If, on the other hand it is 1, then:
%\beq
%\Rp_{c_{pqrs},P-P} (\gv) \equiv -\pd{c_{ijkl}}{c_{pqrs}} g_i g_j g_k g_l = \Tp_{\sdot} (\gv).
%\eeq
%For non-monoclinic parameters the number of 3s is different by mod 2 let us assume that one and only one of p and q is equal to 3, then:
%\beq
%\Rp_{c_{pqrs},P-P} (\gv) \equiv -2 \pd{c_{pqkl}}{c_{pqrs}} g_p g_q g_k g_l + 2 \pd{c_{klpq}}{c_{pqrs}} g_p g_q g_k g_l= 0.
%\eeq

%\section{Incident wavefield}
%%\cite{walter2007} analyzed the empirical relations between the amplitudes of $P$ 
%%and $S$ waves emitted by explosive and double-couple earthquake sources, and came to the 
%%conclusion that $P$ waves dominate the explosions, while $S$ waves dominate 
%%the earthquakes. 
%For simplicity, we consider incident plane waves with unit amplitudes:
%\beq
%\Uv_0 = \Av_0 \exp(i \kv_s \cdot \xv),
%\eeq
%where $\Av_0$ and $\kv_s$ represent the polarization and wavenumber vectors, respectively, of an incident wave.
%For $P$ waves,
%\beq \label{eq:P_sou}
%\Av_0 = \sv,~ \kv_s=\frac{\omega}{\v_p}\sv, \sv = \frac{\xv_s-\xv}{|\xv-\xv_s|}.
%\eeq
%For $SV$ waves,
%\beq
%\Av_0 = \sv_\theta,~ \kv_s=\frac{\omega}{\v_s}\sv,~
%\sv_\theta =  \left( \frac{\gv \times \ev_z}{|\gv \times \ev_z|}\times \gv
%\right).
%\eeq
%For $SH$ waves,
%\beq
%\Av_0 = \sv_\phi,~ \kv_s=\frac{\omega}{\v_s}\sv,~
%\sv_\phi = \left( \frac{\sv \times \ev_z}{|\sv \times \ev_z|}\right).
%\eeq


 

