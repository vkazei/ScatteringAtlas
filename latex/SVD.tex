\section{Spectral sensitivities and numerical analysis of tradeoffs}
In the case of an isotropic background and a general anisotropic scatterer, there are six 
combinations of incident-scattered wave types and 22 elastic 
parameters to perturb. Thus, the total number of radiation patterns specified by 
$\sdot$ is 22*6=132, and this is the amount of data we need to analyze triclinic perturbations. In this section, we present the spectral sensitivities for all these 
parameters and discuss some features of the scattering modes case-by-case 
in the subsections. 

For the orthorhombic and VTI parameters, 
we find all the tradeoffs numerically with SVD and confirm them 
analytically.
Our SVD analysis of the radiation patterns answers the following 
questions:
%\rmrk{fix lists}
\begin{enumerate}
	\item What is the null-space of the linearized dynamic inversions?
	
	\item How many orthorhombic parameters can be resolved from a 
	given scattering mode?
\end{enumerate}
Of course, the answers to these questions depend on the method of acquisition and the signal-to-noise ratio of the data, assuming that coverage is perfect and there is no 
noise. Therefore, the tradeoffs we find are irresolvable.
% and other parameters coupling may exist in realistic illumination conditions.
%only and thus consider the following representative cases for the data components:
%\begin{enumerate}
%	\item $P-$waves only -- first arrivals.
%   \rmrk{next items in this list are removed from the story to simplify it}
%	\item $P$ and $P-SV$ waves.
%	\item $P-P,SV,SH$ -- full-waveform inversion.	
%\end{enumerate} 