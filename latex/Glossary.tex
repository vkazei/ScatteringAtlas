\section*{Glossary}

$\omega$ -- angular frequency;
\\
$\xv$ -- imaging point vector;
\\
$\xv_s, \xv_g$ -- source and geophone radius vectors, respectively;
\\
$\sv$ -- unit vector pointing towards the source along the source wavepath direction;
\\
$\gv$ -- same as $\sv$ but towards the geophone;
\\
$\sp$, $\gp$ -- polarization vectors for the incident and scattered waves respectively
\\
$G_{ik}(\xv,\xv') \equiv G_{ik}(\xv,\xv',\omega)$ -- Green's tensor in the background medium for frequency $\omega$;
\\
$\Kv$ -- wavenumber domain vector;
\\
$\uv(\xv_s,\xv_g)$ -- monochromatic displacement wavefield at frequency $\omega$;
\\
$c_{ijkl}, \cv$ -- stiffness tensor;
\\
$C_{ij}, \Cv$ -- stiffness matrix in Voigt notation; 
\\
\rmrk{HIERARCHICAL PARAMETERIZATION}
\begin{align} \label{eq:jw}
C_{11} &=\rho V^2_{p},
\\
C_{22} &=(1+2\eps_d)\rho V^2_{p},
\\
C_{33} &= \frac{1}{1+2\eps_1} \rho V^2_{p},
\\
C_{12} &= \rho \sqrt{( V^2_{p}-(1+2\gamma_1) V^2_{s})}\\
\times&\sqrt{((1+2\delta_3)V^2_{p}-(1+2\gamma_1) V^2_{s})}\\
- & (1+2\gamma_1 )\rho V^2_{s},
\\
C_{13} &= \rho \sqrt{(\frac{V^2_{p}}{1+2\eps_1}-V^2_{s})
	(\frac{V^2_{p}}{1+2\eta_1}-V^2_{s})}
\\
-&\rho V^2_{s},
\\
C_{23} &= \rho [(\frac{V^2_{p}}{1+2\eps_1}(1+2\gamma_d)\v_s^2)
\\
\times & (\frac{V^2_{p}(1+\eps_d)}{(1+2\eta_1)(1+2\eta_d)}-(1+2\gamma_d)V^2_{s})]^{1/2} 
\\
-&(1+2\gamma_d)\rho V^2_{s},
\\
C_{44} &=(1+2\gamma_d)\rho V^2_{s},
\\
C_{55} &=\rho V^2_{s},
\\
C_{66} &=(1+2\gamma_1)\rho V^2_{s}.
\end{align}
\\
$\rho, \lambda$ -- density and first lamé parameter;
\\
$\delta f$ -- variational perturbation of a function $f$;
\\
$\hat{f}$ -- Fourier transform of a function $f$;
\\
$\sdot$ -- incident and scattered wave types and the perturbation parameter type  $\sdot = \text{P-P}, \lambda$ or $\sdot = \text{P-SH}, C_{13}$;
\\
$\Rp_\sdot(\sv,\gv)$ -- radiation pattern for plane waves and parameter perturbation of type $\sdot$;
\\
$(\sv \tmul \gv)_{ij} \equiv s_i g_j$ -- the outer (dyadic) product of tensors (vectors); 
\\
$:$ -- convolution of tensors on two indexes, e.g.,
$\cv:\sv\gv \equiv c_{ijkl}s_kg_l.$;
\\
Einstein summation is assumed throughout the paper.
\\
We choose the Fourier convention following \cite{hudson1981}:
\beq 
u(t) = \inty \exp(-i\omega t) U(\omega) \d \omega,~ \hat{U}(\Kv) = \intyV \exp(-i\Kv\xv) U (\xv) \d \xv.
\eeq
Natural units convention and equivalent normalization: 
\\ 
scattering should depend only on geometric properties of the elasticity tensor, therefore we can normalize everything by $C_{11}$ in the background, applying the same to the density of the background. The latter is equivalent to the following assumptions:
$\v_p = 1, ~ \rho = 1$.
Slownesses are made dimensionless by multiplying by $\v_{0p}$:
\beq
\svn = \v_{0p} * \sv,
\eeq
densities are normalized by $\rho_0$
\beq
\nmz{\rho} = \frac{\rho}{\rho_0},
\eeq
therefore elastic constants $C_{ij}$ and components of elastic tensor $c_{ijkl}$ are normalized by $\rho_0 \v^2_{0p}$.
$\v_{0p}$ is the P wave velocity in the background media throughout the paper.


 






