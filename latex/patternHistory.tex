\subsection{Brief history of scattering radiation patterns}

%\rmrk{VT: Section 2 should start here}
We consider a multiparameter inverse problem of perfect illumination, where a small anisotropic scatterer is illuminated from all directions. We do not know the scatterer, but we have recorded a certain scattered wave mode, e.g., $P-P$ scattered waves.
% 
The function that depends on incident and scattered wavefields directions and describes the ratio between scattered and incident wavefield amplitudes is the scattering radiation pattern \citep{wu1985,alkhalifah2014}. 
%,podgornova,kamath2016,masmoudi2016,juwon2016,kazei2017,kazei2018}. %  
Using these radiation patterns, which are known from the recorded wave mode, we try to infer the scattering parameter of the scatterer.
%



%
% what has been done?
%
%Evaluation of the scattering radiation patterns through the Born approximation is 
%natural, because the Born approximation represents the leading term of the scattering series and is at the heart of modern seismic 
%imaging techniques such as waveform tomography, FWI and LSRTM.
%
%\subsubsection{In isotropic media}


Different types of radiation patterns indicate different types of elastic 
parameter perturbations and, therefore, can help to distinguish among the parameters from seismic data. \cite{wu1985} described radiation patterns for isotropic elastic 
parameters in the standard ($\lambda,\mu,\rho$) parameterization. 
%
\cite{tarantola1986} considered scattering in several parameterizations of 
isotropic elastic media. He came to the conclusion that a parameterization featuring density and both $P$ and $S$ wave impedances was the 
most promising for reflection seismic 
data. 
%
%\subsubsection{In anisotropic media}
\cite{eaton1994} introduced a scattering function that 
described the scattering of arbitrary plane waves on an arbitrary scatterer in background media with arbitrary anisotropy. 
\cite{hoop1999} focused on the resolvable 
parameters in orthorhombic media and were probably the first to implicitly use 
reflection-based radiation patterns 
\citep{gholami20131,alkhalifah2014,kamath2016,plessix2016}. \cite{juwon2016} 
analyzed 
scattering for several parameterizations of orthorhombic media and proposed a 
hierarchical parameterization, which we use throughout this paper, to reduce the tradeoffs between parameters.