

We note that the right-hand sides of equations \eqref{eq:UPPInt1}, \eqref{eq:UPSVInt}, and \eqref{eq:USHSHInt} are essentially Fourier transforms of the perturbation. Thus, we can generalize these equations as 
\beq \label{eq:KvMain}
\delta U_{WI-WS}(\sv,\gv,\omega) \propto \spv\cdot\gpv \delta \hat{\rho}(\Kv_{WI-WS}(\sv,\gv,\omega)) + \sv\spv : \delta \hat{\cv}(\Kv_{WI-WS}(\sv,\gv,\omega)) :\gv\gpv.
\eeq
Equation~\eqref{eq:KvMain} shows that for a given scattering mode, the wavenumber resolved by inversion depends only on the frequency, source and receiver directions. On the other hand, we can infer all possible directions towards the source and receiver for a given wavenumber. Analyzing scattering for these directions leads directly to irresolvable tradeoffs in inversion at a particular length.

Vertical variations are often dominant in subsurface structures in exploration seismology; therefore, we focus on the vertical wavenumbers $\Kv_{WI-WS}$ in the next section.

%In single mode scattering considerations, the coefficient $\varkappa^{-3}$ can 
%be dropped \citep{podgornova}. 

%but in order to consider the information 
%available from several wave types together we need to take it into account:
%\beq \label{eq:alphaPPPS}
%|\alpha_{PS}| = \varkappa^{-3}|\alpha_{PP}|
%\eeq
%\beq
%\label{eq:incident}
%\kv_g = \frac{\omega}{v_{WS}}\gv,  \kv_s = \frac{\omega}{v_{WI}}\sv.
%\eeq

\subsection{Vertical wavenumber sensitivity}

In this section, we construct sensitivities for vertical wavenumbers $\Kv = (0,0,K_z)^T$.
%
For a given frequency $\omega$ and a particular scattering mode, we can find the corresponding source and receiver directions and remap the radiation patterns into the normalized wavenumber-angle domain using 

\beq \label{eq:sensDef}
\Sp_{WI-WS,\delta \mv} (k_z, \varphi) \equiv \Dp_{\delta \mv, WI-WS} (\sv(k_z,\varphi,WI-WS), \gv(k_z,\varphi,WI-WS)),~
\eeq
where
\beq
k_z = \frac{K_z}{k_0},~
k_{0} = \frac{\omega}{\v_s}. 
\eeq
Equation (44) defines the spectral sensitivity for parameter 
$\delta \mv$. We normalize all the wavenumbers, dividing them by $\omega/V_s$. %acknowledging that the sensitivities scale linearly with 
%frequency in homogeneous backgrounds \citep{kazei2013gp}.
%We focus on the inversion for vertical wavenumbers $\kv_{PP}=(0,0,k_z)$, or horizontally layered structures, which are dominant in most geological scenarios, including those in the global scale, returning us to the remapped reflection-based radiation patterns in a more convenient representation for resolution analysis domain. 
These remapped patterns are functions of two variables ($k_z,\phi$) and, therefore, can be easily plotted and examined. 

\subsubsection{Snell's law in wavenumber domain}
For $S-S$ scattering, the vectors pointing towards the source and the receiver are easily found for a given $k_z\in [0,2]$:
\beq
s_1=-g_1=\frac{1}{V_s}\sqrt{1-\frac{k_z^2}{4}}\cos\phi,~ s_2=-g_2=\frac{1}{V_s}\sqrt{1-\frac{k_z^2}{4}}\sin\phi,~
s_3=g_3=\frac{1}{V_s} k_z/2.  
\eeq
For $P-P$ scattering on a horizontal reflector, the vectors pointing towards the source and the receiver are easily found for a given $k_z\in [0,2\varkappa]$:
\beq
s_1=-g_1=\frac{1}{V_p} \sqrt{1-\frac{k_z^2}{4 \varkappa^2}}\cos\phi,~ s_2=-g_2=\frac{1}{V_p} \sqrt{1-\frac{k_z^2}{4 \varkappa^2}}\sin\phi,~
s_3=g_3= \frac{1}{V_p} \frac{k_z}{2 \varkappa}.  
\eeq
The reflection-based radiation pattern for $C_{55}$ and $P-P$ scattering is represented in \figref{pattern_PP_reflZ}.

For $P-S$ scattering, Snell's law:
\beq
\sv +\gv = \Kv_z
\eeq
leads to the following expressions:
\beq
s_{xy} = \frac{1}{V_s} \sqrt{2\varkappa^2 k_z^2-\varkappa^4+2\varkappa^2-k_z^4+2k_z^2-1}/(2k_z),
\\
s_1=-g_1=s_{xy}\cos(\phi) ,~ s_2=-g_2=s_{xy}\sin(\phi), 
\\
s_3=\frac{1}{V_p}(\varkappa^2+k_z^2-1)/(2\varkappa k_z), g_3 = \frac{1}{V_s}(k_z^2-\varkappa^2+1)/(2k_z),
\eeq
where $\varkappa = \frac{\v_s}{\v_p}$ is the ratio of the shear and longitudinal velocities. 
We show all the possible reflection-based radiation patterns for other cases mapped from the $\phi,\theta$ domain into the wavenumber domain ($\phi,k_z$) as spectral sensitivities in order to directly access the resolution of inversion the normalized wavenumber domain.
%
This mapping allows us to compare wavenumbers and resolution covered by different wave types.

%
In the next section, we analyze the scattering of all the elastic 
constants and find the null-space of the inversion for each scattering mode.
 
\dplot{patterns/C_55/PP_reflZ}{PP_Full/PP_C_55}{(a) The reflection-based radiation pattern for $C_{55}$ and (b) its remapping into the spectral domain. The normalized wavenumber $||\Kv|| \propto \cos \theta_{max}$. In all the patterns, the amplitude sign is indicated by color: red corresponds to a positive reflection coefficient, i.e., no phase shift after scattering; blue shows a negative reflection or scattering coefficient.}
