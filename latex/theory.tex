\subsection{Spectral sensitivities}

Vertical variations are often dominant in subsurface structures. Therefore, we focus on the vertical wavenumbers $\Kv_{WI-WS}$ in the perturbation spectrum. Given the scattering mode, the azimuth, the signal frequency and the wavenumber, \eqrf{K} can be solved for the source and receiver directions $\sv$ and $\gv$, respectively (explicit equations for $\sv$ and $\gv$ are provided in \cref{sec:SnellK}). This means that for a single frequency and single scattering mode a wavenumber defines particular source and receiver directions, that are necessary to illuminate. In other words scattered data can be remapped into the model wavenumber, and azimuth $\phi$ domain:
\beq \label{eq:DataK}
\delta U_{WI-WS}(\Kv,\phi,\omega) \equiv \delta U_{WI-WS}(\sv(\Kv,\phi),\gv(\Kv,\phi),\omega) \propto \delta \mv (\Kv_{WI-WS}).
\eeq
The coefficient of proportionality between model and data in \eqrf{DataK}, which depends on the scattering parameter type, normalized vertical wavenumber and azimuth we call spectral sensitivity. Spectral sensitivity determines how the data represents scattering of a particular parameter and which scales of the parameter are retrievable.

First, we formally define the spectral sensitivities by remapping the reflection-based radiation patterns into the azimuth, vertical wavenumber domain. Then, we demonstrate how to read the spectral sensitivity maps on examples of scattering from a density perturbation, which is an isotropic parameter. As additional example we show scattering form a perturbation in $C_{55}$ which is an orthorhombic parameter.

\subsubsection{Definition}
We construct sensitivities for vertical wavenumbers $\Kv = (0,0,K_z)^T$.
%
For a given frequency $\omega$ and a particular scattering mode, we can find the corresponding source and receiver directions and remap the radiation patterns into the normalized wavenumber-angle domain using 

\beq \label{eq:sensDef}
\Sp_{WI-WS,\delta \mv} (k_z, \varphi) \equiv \Dp_{\delta \mv, WI-WS} (\sv(k_z,\varphi,WI-WS), \gv(k_z,\varphi,WI-WS)),~
\eeq
where
\beq
k_z = \frac{K_z}{k_0},~
k_{0} = \frac{\omega}{\v_s}. 
\eeq
Equation~\eqref{eq:sensDef} defines the spectral sensitivity for parameter 
$\delta \mv$. We normalize all the wavenumbers, dividing them by $\omega/V_s$. %acknowledging that the sensitivities scale linearly with 
%frequency in homogeneous backgrounds \citep{kazei2013gp}.
%We focus on the inversion for vertical wavenumbers $\kv_{PP}=(0,0,k_z)$, or horizontally layered structures, which are dominant in most geological scenarios, including those in the global scale, returning us to the remapped reflection-based radiation patterns in a more convenient representation for resolution analysis domain. 
These remapped patterns are functions of two variables ($k_z,\phi$) and, therefore, can be easily plotted and examined. We illustrate how the reflection patterns are mapped into the wavenumber domain, starting with the spectral sensitivities for density in \figref{explainK_Full}. 

\subsubsection{Examples}
Density is an isotropic parameter; the scattering does not vary with the azimuth in any of the scattering cases. In the $P-P$ scattering mode, the intermediate wavenumbers of density are not illuminated (white line in the middle of \figref{PP_Full/PPdensity_only}), which leads to a poor reconstruction of the density in acoustic FWI \citep{mulder2008}. $P-SV$ waves illuminate the missing intermediate wavenumbers in the density perturbations (\figref{PSV/PSVdensity_only}) and can help reconstruct it. $SH$ waves are difficult to emit in seismic exploration, but they evenly illuminate the full density spectrum (\figref{SHSH/SHSHdensity_only}).

\fplot{explainK}{PP_Full/PPdensity_only}{PSV/PSVdensity_only}{SHSH/SHSHdensity_only}{Scattering of different wave types. (a) Wavenumber/resolution limitations for different types of scattering on the same scale. (b),(c),(d) Scattering in the spectral domain for density for $P-P$, $P-SV$, $SH-SH$ waves. $P$ waves cannot resolve intermediate wavenumbers in the spectrum of density -- see the white stripe in the middle of (b); $P-SV$ waves can close this gap (c). Finally, $SH-SH$ waves can in principle illuminate the whole spectrum of density.}
 
\dplot{patterns/C_55/PP_reflZ}{PP_Full/PP_C_55}{(a) The reflection-based radiation pattern for $C_{55}$ and (b) its remapping into the spectral domain. The normalized wavenumber $||\Kv|| \propto \cos \theta_{max}$. In all the patterns, the amplitude sign is indicated by color: red corresponds to a positive reflection coefficient, i.e., no phase flip after scattering; blue shows a negative reflection or scattering coefficient.}

For anisotropic parameters scattering can depend on the azimuth of the incident and scattered wavefields. The reflection-based radiation pattern for $C_{55}$ and $P-P$ scattering is represented in \figref{patterns/C_55/PP_reflZ}.
%
This mapping allows us to compare wavenumbers and resolution covered by different wave types.
%
In the next section, we analyze the scattering of all the elastic 
constants and find the null-space of the inversion for each scattering mode.