\subsection{Far-field plane-wave scattering}
Radiation patterns simply represent different types of scattered wavefields for a point scatterer 
when the far-field or plane-wave approximation is applied to \eqrf{hudson}. While the derivation is well-known, we return to it in order to generalize the radiation patterns to arbitrary scatterers that 
satisfy \eqrf{hudson}. First, we utilize the approximation from the receiver 
side of \eqrf{hudson}, since it includes Green's tensors and not incident 
wavefields. Then, we treat the source wavefields in the same manner.

\subsubsection{The Born approximation}
We consider an incident wavefield $\Uv^0$ on a perturbation of the density ($\delta\rho(\xv)$) and stiffness tensor ($\delta c_{kjpq}(\xv)$) parameters.
The frequency-domain Born approximation for the scattered wavefield $\delta \Uv$ in an anisotropic elastic medium is \citep[e.g.,][]{hudson1981,beylkin1990,shaw2004}
\beq \label{eq:hudson}
\delta U_i (\xv_s,\xv_\gv,\omega) = \intyV  (\delta \rho(\xv) U^0_k \omega^2 - 
\delta c_{kjpq}(\xv) \pd{U^0_p}{x_q}  \pd{}{x_j})G_{ik}(\xv_g, \xv) \d \xv,
\eeq
where $\omega$ is the angular frequency, $\xv_s$ and $\xv_g$ are the coordinates of the source and receiver, respectively, and $G_{ik}(\xv_g, \xv)$ is the Green's tensor for the background medium. Integration is performed over the volume of the perturbation $\delta \mv = (\delta \rho, \delta c_{ijpq})$, where $\mv$ is the vector of unperturbed parameters. 
We can vectorize equation~\eqref{eq:hudson},
\beq \label{eq:hudsonVec}
\delta \Uv (\xv_s,\xv_\gv,\omega) = \intyV  (\omega^2 \delta \rho(\xv) \Uv^0 (\xv) \cdot\Gv (\xv) - 
\nabla \Gv (\xv) : \delta \cv(\xv) : (\nabla \Uv^0 (\xv))   \d \xv,
\eeq
which already gives us some insight into the scattering. For example, scattering on the density perturbation is determined by the kernel 
\beq
\omega^2 \delta \rho(\xv) \Uv^0 (\xv) \cdot\Gv (\xv),
\eeq 
and, hence, depends only on the amplitudes and polarization of the incident $\Uv^0$ and scattered $\Gv$ wavefields. For instance, if we consider the scattering of $SH$ waves in a vertical plane from a perturbation of density, then the amplitude of the scattered wavefield will be constant.
%\figref{SHSH/SHSHCij}. 
At the same time, the scattering from perturbations of elastic constants $C_{ij}$, which are driven by the kernel
\beq
\nabla \Gv (\xv) : \delta \cv(\xv) : (\nabla \Uv^0 (\xv)),
\eeq
can also depend on the propagation directions
of the incident and scattered wavefields, which are extracted by the gradient $\nabla$ operator.
 
\subsubsection{Wavefields scattered by perturbation}
The Green's tensor $G_{ik}$ in the Born approximation~\eqref{eq:hudsonVec} physically 
takes care of the propagation of a scattered wavefield from the perturbation to the 
receiver.
The far-field approximation for the Green's tensor \citep[e.g.,][]{snieder2002} in a homogeneous, isotropic background 
%is
%\begin{align} \label{eq:farField}
%	\Gv(\xv,\xv_g) &= \alpha\bigg(  e^{i\frac{\omega}{\v_p}||\xv - \xv_g||} \gv\gv
%	\\
%	&+\frac{1}{\varkappa^2} e^{i\frac{\omega}{\v_s}||\xv - \xv_g||} (\gv_{\theta}\gv_{\theta} + \gv_{\phi}\gv_{\phi})\bigg), 
%	\varkappa = \frac{\v_s}{\v_p}, \alpha = \frac{1}{4\pi\rho \v_p^2||\xv - \xv_g||},
%\end{align}
%where $\gv,~\gv_{\theta},$ and $~\gv_{\phi}$ are the polarization vectors of the $P$, $SV$, and $SH$ waves, respectively. Thus, the far-field part of the Green's tensor (26) 
consists of $P,~ SV,$ and $SH$ waves:
\beq \label{eq:3G}
\Gv(\xv_g, \xv) = \alpha (\Gv_P(\xv_g, \xv) + \Gv_{SV}(\xv_g, \xv) + \Gv_{SH}(\xv_g, \xv)). 
\eeq
While $SV$ and $SH$ waves in isotropic media most often come together, they are usually recorded separately and therefore can be analyzed separately too. 
The coefficient 
\beq
\alpha = \frac{1}{4\pi\rho \v_p^2||\xv - \xv_g||}.
\eeq
is dropped in our resolution analysis, since it is related to geometrical spreading and can be effectively compensated using pseudo-hessian techniques. Different components of the Green's function are approximated locally by plane waves. For the first component $\Gv_P$, the approximation leads to
\beq \label{eq:GvP}
\Gv_P = e^{i\frac{\omega}{\v_p}||\xv_g- \xv ||} \gvn\gvn \simeq 
e^{i\frac{\omega}{\v_p}\gv\cdot(\xv - \xv_g)}\gvn\gvn, ~ 
\gvn = \frac{\gv}{|\gv|} = \frac{\xv-\xv_g}{|\xv-\xv_g|},
\eeq
which is the $P$ wave approximated by a plane wave at the location $\xv$. ($\gv\gv)_{ik} = g_i g_k$ denotes standard outer product of vector $\gv$ with itself. 
%For the second and third 
%components, $\Gv_{SV}$ and $\Gv_{SH}$, similar expressions can be provided 
%\citep{snieder2002}.
The second component, $\Gv_{SV}$, represents a shear wave polarized in the vertical plane 
that contains $\xv$ and $\xv_g$:
\beq \label{eq:GvSV}
\Gv_{SV} = \frac{1}{\varkappa^2} e^{i\frac{\omega}{\v_s}||\xv - \xv_g||}\gv_{\theta}\gv_{\theta} \simeq 
\frac{1}{\varkappa^2} e^{i\frac{\omega}{\v_s}\gv\cdot(\xv - \xv_g)}\gv_{\theta}\gv_{\theta},~
\gv_\theta = \left( \frac{\gv \times \ev_z}{|\gv \times \ev_z|}\times \gvn \right). 
\eeq
Finally, the third component of the Green's tensor, $\Gv_{SH}$, is an $SH$ wave polarized along $\gv_\phi$:
\beq \label{eq:GvSH}
\Gv_{SH} = \frac{1}{\varkappa^2} e^{i\frac{\omega}{\v_s}||\xv - \xv_g||}\gv_{\phi}\gv_{\phi} \simeq 
\frac{1}{\varkappa^2} e^{i\frac{\omega}{\v_s}\gv\cdot(\xv - \xv_g)}\gv_{\phi}\gv_{\phi},
\gv_\phi = \left( \frac{\gv \times \ev_z}{|\gv \times \ev_z|}\right),
\eeq
which is the horizontal component of a shear wave.
%\subsubsection{Incident wavefield}

For the incident wavefield we consider point forces in three directions $\svn, \sv_\phi, \sv_\theta$, leading to three respective wavetypes:
\beq \label{eq:P_sou}
\Uv_{0P} = \Gv_P(\xv_s,\xv) \cdot \svn,
\eeq
\beq
\Uv_{0SV} = \Gv_{SV}(\xv_s,\xv) \cdot \sv_\theta,
\eeq
\beq
\Uv_{0SH} = \Gv_{SH}(\xv_s,\xv) \cdot \sv_\phi.
\eeq
%\cite{walter2007} analyzed the empirical relations between the amplitudes of $P$ 
%and $S$ waves emitted by explosive sources and double-couple earthquake sources, and came to the conclusion that $P$ waves dominate the explosions, while $S$ waves dominate the earthquakes.

%For simplicity, we consider incident plane waves with unit amplitudes:
%\beq
%\Uv_0 = \Av_0 \exp(i \kv_s \cdot \xv),
%\eeq
%where $\Av_0$ and $\kv_s$ represent the polarization and wavenumber vectors, respectively, of an incident wave.
%For $P$ waves,
%\beq \label{eq:P_sou}
%\Av_0 = \svn,~ \kv_s=\frac{\omega}{\v_p}\svn, \svn = \frac{\xv_s-\xv}{|\xv-\xv_s|}.
%\eeq
%For $SV$ waves,
%\beq
%\Av_0 = \sv_\theta,~ \kv_s=\frac{\omega}{\v_s}\svn,~
%\sv_\theta =  \left( \frac{\sv \times \ev_z}{|\sv \times \ev_z|}\times \svn
%\right).
%\eeq
%For $SH$ waves,
%\beq
%\Av_0 = \sv_\phi,~ \kv_s=\frac{\omega}{\v_s}\svn,~
%\sv_\phi = \left( \frac{\sv \times \ev_z}{|\sv \times \ev_z|}\right).
%\eeq

