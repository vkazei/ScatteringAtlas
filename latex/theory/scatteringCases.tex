\subsection{Model wavenumbers and scattering modes}
We start with scattered $P$ waves as they are the easiest to handle and by far the most popular in FWI applications. By substituting equations \eqref{eq:P_sou} and \eqref{eq:GvP} into equation \eqref{eq:hudson}, and projecting the scattered wavefield $\Uv$ onto the direction of $\gv$ (from the perturbation to the receiver), we obtain the approximate amplitude of the scattered $P-P$ waves \citep{kazei2018} as
\beq \label{eq:UPPInt}\label{eq:UPPInt1}
\delta U_{PP} = \delta \Uv_P \cdot \svn  \propto 
%\frac{\delta \mathbf{U} \cdot \gv}{\alpha_{PP}} \simeq
\intyV   e^{i\Kv \cdot \xv} (\svn\cdot\gvn \delta \nmz{\rho} (\xv) + \svn\svn : \delta 
\nmz{\cv}(\xv) :\gvn\gvn) \d \xv, 
\eeq
where
%\beq 
%\alpha_{PP} = \frac{\omega^2}{v^2_p} \alpha_s \alpha_g e^{i\frac{\omega}{\v_p}(-\xv_s\sv-\xv_g\gv)}
%\eeq
%and
\beq \label{eq:K}
\Kv = \omega(\sv+\gv).
\eeq

Analogously, for the amplitude of the scattered $P-SV$ waves, we use equations \eqref{eq:P_sou} and \eqref{eq:GvSV}, and project \eqref{eq:hudson} onto vector $\gv_{\theta}$ to obtain
\beq \label{eq:UPSVInt}
\delta U_{PSV} = \delta \Uv_P \cdot \sv_\theta 
\propto   
%\frac{\delta \mathbf{U} \cdot \gv_\theta}{\alpha_{PS}} \simeq\frac{1}{\varkappa^3}
\intyV  e^{i\Kv_{PS}\cdot\xv} (\svn\cdot\gv_{\theta} \delta \nmz{\rho} + \frac{1}{\varkappa} \svn\svn : \delta \nmz{\cv} :\gvn\gv_{\theta}) \d \xv. 
\eeq
%where
%\beq
%\alpha_{PS} = \frac{\alpha_{PP}}{\varkappa^3} e^{i\omega\gv_\theta \cdot (\frac{\xv_g}{\v_p}-\frac{\xv_g}{\v_s})}, \Kv_{PS}=\frac{\omega}{\v_p}\sv+\frac{\omega}{\v_s}\gv.
%\eeq

Replacing the polarization vector $\gv_\theta$ in (39) with $\gv_\phi$ is sufficient to obtain similar expressions for the $P-SH$ scattering scenario. For $S$ waves, we obtain similar expressions by using equations \eqref{eq:P_sou}, \eqref{eq:GvSH}, and \eqref{eq:hudson}:
\beq \label{eq:USHSHInt}
\delta U_{SHSH} = \delta \Uv_{SH} \cdot {\sv_\phi} 
\propto 
%\equiv  \frac{\delta \mathbf{U} \cdot \gv_\phi}{\alpha_{SS}} \simeq \frac{1}{\varkappa^3}
\intyV  e^{i\Kv_{SS}\cdot\xv} (\sv_\phi \cdot \gv_\phi \delta \rho + \frac{1}{\varkappa^2} \nmz{\sv}\sv_\phi : \delta \nmz{\cv} :\nmz{\gv}\gv_{\phi}) \d \xv. 
\eeq
%where
%\beq
%\alpha_{SS} = \frac{\alpha_{PP}}{\varkappa^6} e^{i\omega\gv \cdot (\frac{\xv_g}{\v_s}-\frac{\xv_g}{\v_s})}, 
%\Kv_{SS}=\frac{\omega}{\v_s}(\sv+\gv).
%\eeq

For different types of scattering, our further resolution analysis is based on equations 
(\ref{eq:UPPInt1}-\ref{eq:USHSHInt}) and their analogs, which describe the monochromatic scattering of $P$ waves.

 

Different scattering modes contribute to different wavenumbers at 
the same reflection angle and frequency of the image. Therefore, in order to investigate all the
information available, we need to remap the reflection-based radiation patterns into 
the spatial wavenumber domain (\figref{K_PP_Full}).

\tplot{K_PP}{K_PS}{K_SS}{For the same incident angle 
	and same temporal frequency different modes provide illumination to different 
	vertical wavenumbers. $P-S$ waves illuminate the lowest wavenumbers (good for FWI), 
	$S-S$ waves -- the highest wavenumbers (good for migration), and $P-S$ waves -- the intermediate wavenumbers.}



