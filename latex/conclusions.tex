\section{Conclusions}
%
We remapped reflection-based radiation patterns into the spectral domain for all the anisotopic elastic parameters, and incident and scattered wave types. Each remapped pattern, or spectral sensitivity, characterizes the resolution at which a particular parameter can be reconstructed. We found some apparent tradeoffs among the triclinic parameters, and showed that only monoclinic parameters can be robustly reconstructed from monotypic waves. Table 1 summarizes the tradeoffs found for monotypic waves. 
%

For the orthorhombic parameters, we performed SVD and found the exact null-spaces for each mode of scattering. The simple intersections of the null-spaces provide information about the null-spaces themselves when several modes are used in the inversion. The intersections of $P-P$ waves, which are used together with $P-S$ conversion, and the tradeoffs among the converted waves are provided in Table 2. From the $P-P$ and $P-SV$ waves, we can reconstruct all the parameters except $\gamma_1$; from $P-SH$ waves, we can theoretically reconstruct all the orthorhombic parameters.

In realistic scenarios, limited apertures, frequency content, and noise in the data may limit the set of recoverable parameters, while prior assumptions introduced via regularization or constraints can expand the set of recoverable parameters. This paper will help researchers choose the optimal constraints, regularizations, and parameterizations when designing elastic anisotropic inversions. 

\begin{table} \label{tab:tradeoffs}
	\begin{tabular}{|c|c | c | c | c}
		\hline
		    Data:     &                                 $P-P$                                  & $SH-SH$                                          & $SV-SV$                                      &  \\ \hline
		 Nulls VTI:   &                   $2\Cv_{12}-\Cv_{66}\equiv\gamma_1$                   & $V_p$                                            & $\Vv_p$                                      &  \\
		              &               $\Cv_{11}+\Cv_{12}+\Cv_{22}-\Cv_{33}+\rho$               & $\eps_1$                                         & $\epsv_1$                                      &  \\
		              & $V_s - 8\varkappa \eta_1 \equiv 2\Cv_{13}+\Cv_{55}+2\Cv_{23}+\Cv_{44}$ & $\eta_1$                                         & $\gammav_1$                                      &  \\ 
		              &                                                                        & $2\varkappa \rho - (1+\varkappa) \Vv_s + 2\gammav_1$ &                                    &  \\ \hline
		 Nulls ORT:   &                                                                        & $\eps_d + \delta_3 \equiv 2\Cv_{22}+\Cv_{12}$    &                                    &  \\
		              &                                                                        & $\Cv_{23}\equiv \eta_d$                  & $ \lambda$                                   &  \\
		              &        $2\Cv_{23}+\Cv_{44} \equiv 8\varkappa \etav_d+ \gamma_d$         &                               & $\epsv_1$                                     &  \\
		              &                                                                        &                             & $\epsv_d \propto \deltav_3$                    &  \\
		              &                                                                        &                                                  &                                              &  \\ \hline
		  \# par.:    &                                   6                                    & 4                                                & 6                                            &  \\
		Set VTI: &                       $V_p$, $\eps_1$, $\eta_1$                        & $V_s$,$\gamma_1$                                 & $\rho,~V_s,~\eta_1$                          &  \\ \hline
		Set ORT: &                       "acoustic, $\rho =const$"                        & $V_s,~\eps_d,\gamma_1,\gamma_d$                  & $\rho,~V_s,~\eps_d,~\eta_1,~\eta_d,\gamma_d$ &  \\ \hline
	\end{tabular}
\caption{Principally irresolvable linear combinations of orthorhombic parameters, number of resolvable parameters, and possible set of parameters that 
	could be resolved from the $C_{ij}, \rho$ parameters for different types of scattering recorded under perfect illumination.}
\end{table}



\begin{table} \label{tab:tradeoffs}
	\begin{tabular}{| c|c | c | c|c |c|}
		\hline
		Data:      &       $P-SV$       & $SV-SH$              & $P-SH$  & $P-P, P-SV$ & $P-P, P-SV, P-SH$ \\ \hline
		Nulls VTI: &     $\gamma_1$     & all VTI              & all VTI & $\gamma_1$  & $\gamma_1$        \\
		           & $V_p\equiv\lambda$ &                      &         &             &                   \\
		Nulls ORT: &         -          & $2\eps_d + \delta_3$ & -       &      -      & -                 \\ \hline
		\# par.:   &         8          & 3                    & 4       &      9      & 9
		\\ \hline
	\end{tabular}
	\caption{Same as Table 1, but for converted waves and their combinations with $P$ waves.}
\end{table}



%\input{tables/OBC} 


